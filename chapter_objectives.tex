\chapter{Functional Objectives and Exisiting Solutions}

\section{Requirements}
TODO: wo kommen die her, kann man da was machen?
\subsection{Functional Requirements}
\begin{itemize}
  \item TODO: usable to all
	\item User Profiles \\
	Anyone can see another user’s profile consisting of basic information about the user such as Name, Location, E-Mail and personal skills. Personal skills are composed of a name, a knowledge level and a will level, both on a scale from one to four.
	\item Provide/Edit skills\\
	Users can add new skills from a pool of known skills to their own profile. Already added skills can be edited and removed from the profile.
	\item Search\\
	A search function can be used to find people who have added one or more specific skills to their profile. When searching for multiple skills, only persons matching all of them will will be displayed.
	\begin{itemize}
		\item Ranking\\
			By default, the search results order should be defined by a score aggregating the individual employee's skill level, will level and grade of spezialization in the searched skills. Employees willing to enhance their knowledge about a searched skill should be preferred to others having the same knowledge but a lower will.
		\item Sorting\\
			The user should be able to sort the search results not only by said score,
			but also solely by knowledge and will level.
	\end{itemize}
	\item Management of known skills\\
	New skills can be added to the set of known skills in the application. Existing skills can be edited and removed. Users personal skills are automatically updated when a skill has been edited so that the integrity of the user profiles is maintained at all times.
\end{itemize}

\subsection{Non Functional Requirements}
TODO: Klären und formulieren
\begin{itemize}
	\item Desktop/Devices
	\item Browsers
	\item Scalability
	\item Load/Response Times
\end{itemize}



\section{Commercial Solutions}

\subsection{Skills Base}
Skills Base\footnote{http://http://www.skills-base.com/} offers the required features, but also includes a large number of functionality SinnerSchrader does not need and ist not willing to use. This includes assessments, the categorization of skills and a role model for advanced access rights configuration.
The search function does not provide searching for multiple skills. Furthermore, the sorting of results found cannot be customized. A central point of the application are dashboards displaying information about the most popular skills in the organisation and long term statistics.
\begin{figure}[!htp]
    \centering
    \includegraphics[width=0.8\textwidth]{images/skillsbase-dashboard.png}
    \caption{SkillsBase Dashboard}
    \label{fig:skillsbase_dashboard}
\end{figure}

\subsection{Talent Management (engage!)}
Talent Management\footnote{http://www.infoniqa.com/hr-software/skill-management} is a module for Infoniqa’s management software engage!. It offers advanced features for managers such as a powerful search function controlled via a special query language. It also includes data about the employees’ salaries, feedback protocols and certificates. It can only be used in combination with engage!, an complete human resources management solution including features like time tracking, e-learning, applicant management and payroll accounting.

\begin{figure}[!htp]
    \centering
    \includegraphics[width=0.8\textwidth]{images/talent_management_-_skillmanagement_-_skillsuche.png}
    \caption{Talent Management Search}
    \label{fig:talent_management}
\end{figure}

\subsection{SkillsDB Pro}
SkillsDB Pro\footnote{http://www.skillsdbpro.com} is an application designed to serve as a database in an organization providing an overview about every person’s own skills and and trainings only to themselves and their manager. The search function is capable of searching for multiple skills combined with different logical operators which enables users to enter very sophisticated queries.
Not only can users provide information about their skills, but managers can also do this with the limitation that no employee can see their manager’s rating about themselves.
Furthermore, only managers can search for persons. Taking into consideration that SinnerSchrader needs a tool to enable everyone to find someone with a specific skillset, this is a serious disadvantage.
SkillsDB also offers features SinnerSchrader does not intent to use including, but not limited to the automatic generation of project reports based on plan succession and demands for assessments.

\subsection{Conclusion}
None of the analyzed softwares offers all features required, but all of them include various functions SinnerSchrader does not intend to use, which brings undesired complexity into the applications.
One of the most critical features, sorting the search results by best match, is not offered by any of the commercial solutions.
Furthermore, all those systems differentiate between employees and their supervisors and thus restrain transparency. Instead of a solution for monitoring employees and rating them, we want a tool for everyone to find another person who offers the skills needed to solve a concrete problem.

\subsubsection{Pain Point Fitness Scoring}
As shown by \cite{CanosDaros2013}, motivation is a vital factor regarding any employee's performance and quality of work.
Although motivation is a complex construct of many highly diverse dimensions, the overlap of a person's interests and their duties is a key aspect to it.
Assuming that every member of the company has some skills they prefer to employ over others, matching people to tasks that require the exact same abilities they are interested in employing will lead to more motivated employees and thus have a positive impact on the overall productivity of SinnerSchrader.
Consequentely, when searching for persons having specific skills, the application should not only take into acount the employees' skills but only their preferences in order not to find the most skilled, but the best fitting one. Unfortunally, none of the examined applications does provide a way to aggregate both, skills and preferences, into a
single score indicating the overall grade suitability of a person relative to the searched skills.
