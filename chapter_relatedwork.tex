\chapter{Related Work}

Skills management is a trending topic in today's management world, as it is a vital part of the success of a modern business.
Darvish et al. highlight the importance of knowledge management in various forms of institutions and compare different knowledge management strategies.
Furthermore, the authors show how enterprises can introduce tools and mindsets to use the positive effects of knowledge management in their organizations \cite{darvish}, but do not discuss concrete software tools.
A market analysis by Lehner, however, compares multiple commercially available software systems for skills management and provides information adapted in \ref{commercial}  \cite{Marktanalyse}.
\newline
Beck outlines a case study that deals with the introduction of a skills management system at \textit{Putzmeister GmbH} and reveals important success factors of such a software, e.g. ways to motivate employees to provide a sufficient amount of data, legal concerns and cooperation with the works council (\textit{Betriebsrat}), obstacles that occur in the maintenance phase of the system's lifecycle, and usability requirements \cite{beck}.
\newline
In contrast to the systems analyzed by Lehner and Beck, the application that this thesis will deal with does not only
show the information saved in its database, but also provides a powerful search function and recommends employees based on the combination of multiple personal factors including their motivation.
\newline
The concept of approaching team building and management challenges with algorithms has been evaluated and successfully implemented multiple times.
In 2013, Ivanovksa et al. compared various data mining algorithms for the automatic composition of teams and propose the usage of \textit{Bayesian Networks} for this kind of problems \cite{ivanovska}. Unfortunately, the authors do not take into account factors like the employees' motiviation and satisfaction. Those aspects have been examined by Canós-Darós who introduced an algorithm to measure employees' motivation \cite{CanosDaros2013} and highlights aspects the algorithm used by the application should include.
Spoonamore et al. created an algorithm that deals with the matching of personnel to open positions in the \textit{United States Navy} \cite{USN}. An adaption of this algorithm lays the foundation for the scoring algorithm used in this application (see \ref{fitscorealg}). Furthermore, the authors describe non-technical requirements an algorithm which ranks personal abilities has to
meet in order to be accepted by the target audience that will be scored by it.
\newline
This thesis does not cover the visual concept and implementation of the applications graphical user interface, since
Strecker's bachelor's thesis addresses this field of functionality \cite{strecker}.
