\chapter{Résumé and Outlook}
In this thesis, the creation of a skills management application custom-tailored to SinnerSchrader has been drafted from its underlying concept to its technical design and its implementation. The motivation for SinnerSchrader to introduce such a system and the company-specific challenges it has to overcome have been outlined.
To compile the requirements for the system, semi-structured interviews with representatives for the groups of stakeholders, namely the Flow Team, project managers, and \textit{regular} employees, have been conducted. Those requirements provide the basis for the analysis of available skill management tools. As shown, two major factors required by SinnerSchrader cannot be served by any of the commercial solutions: the tool is supposed to put emphasis on collaboration, not supervision, and users should be able to search for persons that not only have knowledge about a certain topic but should also take into account their interests and personal preferences. These two factors form the backbone of the application's technical design; its central feature is a search function that finds the best matching employees for the searched skill set. The scoring algorithm determining how well a person matches the search query has been designed, implemented, and evaluated. Furthermore, the construction and implementation of recommender systems that enrich the user experience by proactively presenting favorable items to them have been laid out.\\
Both the implementation and the underlying concept have been evaluated regarding possible concerns including technical issues and the fulfilling of the end users' needs. To examine the latter, a survey has been conducted and analyzed.\\
Altough the application has been tested using a fair amount of generated data, live data entered by real users could invalidate the results found in the evaluation phase and reveal obstacles not considered in its design. Only a long-run test phase exposing the application to the users will provide sufficient information about those factors. Running such a test requires a graphical component to present the user with an interface; this will be subject to further research.
The constructed tests, however, suggest that the novel approach of using a scoring algorithm in the context of skills management provides a solid basis to find the
most suitable employee.\\
Although the implemented backend provides the required features, future development iterations will bring new functionality. Conceivable extensions include categories for skills, a process to manage certifications, or the possibility to customize the weighting parameters of the fitness score algorithm in the frontend.\\
The current state of the application, however, already shows the potential to become an effective tool for collaboration and skills management at SinnerSchrader.
