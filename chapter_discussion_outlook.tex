\chapter{Résumé and Outlook}
In this thesis, the creation of a skills management application custom-tailored to SinnerSchrader has been drafted from its underlying concept to its partial implementation. The motivation why SinnerSchrader needs such an application as well as the challenges specific to this company have been outlined. Based on this information, requirements the tool will have to fulfill have been compiled; they provide the basis for the analysis of available skill management tools. As shown, two mayor factors required by SinnerSchrader cannot be served by any of the commercial solutions: the tool is supposed to put emphasis on collaboration, not supervision, and users should be able to search for persons that not only have knowledge about a certain topic but should also take into account their interests and personal preferences.
These two aspects form the backbone of the application's design including an algorithm that represents them in the search function and thus is an essential part the tool's proper working.
Furthermore its technical structure has been laid out and the backend component dealing with its business logic has been implemented.
Both the implementation and the underlying concept have been evaluated regarding possible concerns including technical issues and the fulfilling of the end users' needs. To examine the latter, a survey has been conducted and analyzed.\\
Altough the application has been tested using a fair amount of generated data, real data entered by real users could invalidate the
results found in the evaluation phase and reveal obstacles not considered in its design. Only a long-run test phase exposing the application to the users will provide sufficient information about those factors. Unfortunately, the frontend of the application is not finished yet, so that that a user test cannot be run.
The constructed tests suggest that the basic principles behind the fitness score algorithm provide a solid foundation for a new type of skills management tools that incorporate not only employees' skills, but also their motivation and preferences thus leading to an increasing employee satisfaction and accelerating personnel development.
Integratig the search algorithm or an approach based on its principles into a more sophisticated and feature-rich application, like the ones available on the market, would be a logical step towards such a management tool. \\
The concrete application created for SinnerSchrader has purposely been designed to be a simple search tool as the focus lays on a streamlined user experience. The implementation fullfils all requirements defined for this special use case and once the frontend component will be integrated into the application, it has the potential to become a deeply useful gadget for the daily work at SinnerSchrader.
